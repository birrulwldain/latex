% \fancyhf{} 
% \fancyfoot[R]{\thepage}

\chapter{PENDAHULUAN}
%\thispagestyle{plain} % Halaman pertama bab menggunakan gaya plain

\section{Latar Belakang}
% Menambahkan lorem ipsum
\par Peningkatan standar hidup menyebabkan peningkatan minat terhadap kesehatan dan lingkungan. Keamanan pangan dan dampak lingkungan menjadi kriteria pemilihan yang penting untuk konsumsi makanan. Persepsi konsumen tentang kualitas makanan dan kepedulian tentang dampak lingkungannya mendorong meningkatnya permintaan akan makananorganik. Penelitian tentang perilaku konsumen dalam konteks makanan organik bertujuan untuk menyoroti berbagai faktor yang mempengaruhi konsumsi. Sebagian besar penelitian telah berusaha menjelaskan peran ekonomi, budaya, sosio-demografis, dan faktor psikologis pada konsumen (Han, 2022).
\par Telur organik diproduksi dengan menghindari antibiotik dan bahan kimia sintetis. Ayam yang memproduksi telur organik harus memiliki akses ke alam bebas dan tidak dapat ditempatkan di dalam kandang. Pakan ayam harus bebas dari antibiotik dan bahan kimia sintetis dan harus mengandung biji-bijian yang berasal dari tanaman yang bersertifikat "organik". Lahan untuk menghasilkan tanaman tersebut harus bebas dari tanaman "rekayasa genetika" dan pupuk sintetis selama 3 tahun atau lebih. Penggunaan antibiotik hanya diperbolehkan untuk wabah penyakit. Ayam liar memenuhi kebutuhan mereka secara alami.
\par Penyebaran dan pemanfaatan ayam domestik secara global telah memperkenalkan persyaratan baru seperti perlindungan dari iklim dan berbagai tingkat pengurungan termasuk perumahan. Kandang memfasilitasi: pengumpulan telur, penangkapan ayam, dan perlindungan dari predator, hama, dan iklim yang merugikan. Sistem pengurungan meliputi: di lapangan/padang rumput di mana iklimnya cukup hangat, dikurung di dalam ruangan di dalam kandang dengan akses ke lapangan, dan dikurung sepenuhnya di dalam ruangan baik di dalam kandang atau kandang dengan satu atau beberapa ekor ayam per kandang. Kandang harus memiliki sistem ventilasi untuk menyediakan udara segar dan untuk menjaga suhu dan kelembaban yang dihasilkan oleh ayam. Ayam petelur menghasilkan banyak panas tubuh (Chambers, 2017).
\par Pertanian organik merupakan salah satu saran konkret, namun kontroversial untuk meningkatkan keberlanjutan sistem pangan. Beberapa penulis berkontribusi pada diskusi tentang hasil panen yang lebih rendah dalam pertanian organik dengan mempertimbagkan ketersedian nutrisi, tetapi tidak satupun dari mereka yang memberikan analisis yang kuat tentang ketersediaan nutrisi dalam sistem produksi pangan. Selain itu, stud-studi ini tidak menggunakan pendekatan sistem pangan terperinci, dan tidak membahas peran rezim pemberian pakan ternak, tren konsumsi, dan pemborosan makanan yang mungkin dimainkan, yang mana merupakan faktor untuk strategi secara substansial dapat mengurangi kebutuhan lahan, sekaligus mengurangi dampak lingkungan serta berkontribusi terhadap ketersedian pangan global (Niggli, 2017). Hal ini kemudian berlaku untuk telur organik yang kualitas produk dengan pakan organik dan produk yang diberi pakan buatan sulit untuk dibedakan dari penampilannya saja, dalam hal ini telur kampung. Karena hampir tidak ada perbedaan tampak antara telur ayam kampung dengan pakan organik dan telur ayam kampung yang diberi pakan buatan, pengujian indikator kualitas yang membosankan dan memakan waktu sulit dilakukan di pasar. Hal ini menyebabkan beberapa produsen yang tidak bertanggung jawab menggunakan pakan buatan dengan produksi yang banyak. Oleh karena itu, sangat penting untuk mengembangkan metode yang dapat dengan cepat dan akurat mengevaluasi kualitas telur ayam kampung secara real time.
\par Laser-Induced Breakdown Spectroscopy (LIBS), juga kadang-kadang disebut Laser-Induced Plasma Spectroscopy (LIPS) atau Laser Spark Spectroscopy (LSS) telah berkembang pesat sebagai teknik analitis selama dua dekade terakhir. Teknik ini menggunakan laser berdenyut berenergi rendah (biasanya puluhan hingga ratusan mJ per denyut nadi) dan lensa pemfokusan untuk menghasilkan plasma yang menguapkan sejumlah kecil sampel. Sebagian cahaya plasma dikumpulkan dan spektrometer menyebarkan cahaya yang dipancarkan oleh spesies atom dan ionik yang tereksitasi dalam plasma, detektor mencatat sinyal emisi, dan elektronik mengambil alih untuk mendigitalkan dan menampilkan hasilnya (Cremers, 2013).
\par Jenis dan komposisi elemen dalam material berdampak pada sifat-sifatnya, baik secara langsung maupun tidak langsung. Sangat penting untuk melakukan analisis unsur untuk mengevaluasi kinerja material. Metode konvensional meliputi spektrometri serapan atom (AAS), spektrometri massa plasma yang digabungkan secara induktif (ICP-MS), spektrometri emisi plasma yang digabungkan secara induktif (ICPAES), spektrometri fluoresensi sinar-X (XRF), spektroskopi serapan laser dioda yang dapat disetel, spektroskopi fotokustik yang disempurnakan dengan kuarsa, spektroskopi fototermal yang disempurnakan dengan kuarsa (QEPTS), dan spektroskopi serapan sisir ganda (dual-comb absorption spectroscopy) [1-3]. Karena pengoperasiannya yang rumit dan prosesnya yang memakan waktu, metode-metode ini biasanya digunakan di laboratorium. Dalam beberapa tahun terakhir, para ilmuwan telah mencari dan mengembangkan uji analitik baru dengan respons yang cepat, pengoperasian yang mudah, dan keandalan yang tinggi. Spektroskopi kerusakan yang diinduksi laser (LIBS) adalah spektrometri atom baru yang menjanjikan, lebih serbaguna daripada metode tradisional . LIBS juga sering disebut sebagai spektroskopi plasma yang diinduksi laser (LIPS) atau spektroskopi percikan laser. Sebagai sumber eksitasi dalam LIBS, sinar laser berdenyut difokuskan ke permukaan sampel dengan menggunakan lensa pemfokusan. Melalui ionisasi multiphoton, atom, ion, dan molekul dalam area fokus laser menyerap energi laser dan menghasilkan elektron bebas awal. Dengan efek bremsstrahlung terbalik, elektron bebas dipercepat oleh medan elektromagnetik sinar laser dan kemudian bertabrakan dengan partikel dalam gas sekitar dan bahan sampel untuk menghasilkan lebih banyak elektron bebas. Elektron bebas yang baru tercipta juga dipercepat oleh medan listrik, menghasilkan proses ionisasi longsoran elektron (EAI) sepanjang durasi pulsa laser . Selama fenomena kerusakan, plasma dihasilkan pada permukaan sampel. Spesies permukaan dapat disimpulkan secara kuantitatif dengan menganalisis spektrum emisi plasma . LIBS telah menjadi teknik yang menarik dan populer di bidang analisis kimia karena keunggulannya yang unik, seperti aplikasinya pada cairan, gas, dan padatan, tidak ada perlakuan awal sampel, deteksi simultan beberapa elemen, dan deteksi jarak jauh tanpa kontak di berbagai bidang, termasuk pembersihan laser, perlindungan lingkungan, eksplorasi ruang angkasa, dan pelestarian warisan budaya (Zhang, 2022).


\section{Rumusan Masalah}
\par Han dkk. menyatakan bahwa standar hidup meningkat dan berjalan dengan seiring kepedulian terhadap kesahatan yang mana difaktori oleh pangan global, dalam hal ini pangan organik. Namun, keaslian dari pangan organik dalam hal ini komoditas telur kampung sulit untuk di analisa dalam waktu singkat dan akurat. Oleh karena itu muncul beberapa pertanyaan meynakut komposisi telur dan konsentrasi unsur pernyusunnya untuk membuktikan keasliannya; bagaimana kadar kandungan non-organik pada telur serta apa yang mempengaruhi keadaan terhadap kemunculan kandungan non-organik pada telur kampung.
%\begin{enumerate}
%	\item lorem ipsum dolor sit amet, consectetur adipiscing elit. Sed euismod, nisl quis lacinia ultricies,
%	\item lorem ipsum dolor sit amet, consectetur adipiscing elit. Sed euismod, nisl quis lacinia ultricies,
%	\item lorem ipsum dolor sit amet, consectetur adipiscing elit. Sed euismod, nisl quis lacinia ultricies,
%	\item lorem ipsum dolor sit amet, consectetur adipiscing elit. Sed euismod, nisl quis lacinia ultricies,
%\end{enumerate}

\section{Tujuan Penelitian}
\par Berdasarkan latar belakang dan rumusan masalah diatas, tujuan dari penelitian ini adalah untuk menganalisis komposisi serta konsentrasi pada telur organik untuk membandingkan dan menetapkan keaslian telur  organik tersebut bebas dari kandungan non-organik berdasarkan perlakuan  pakan sampel menggunakan metode CF-LIBS.

\section{Manfaat Penelitian}
\par Dengan dibuktikan komposisi dan konsentrasi pada sampel, akan memberikan kepedulian masyarakat tentang keaslian telur organik. Kesadaran akan pakan yang diberikan dan metode pembuktian keaslian akan meningkatkan nilai jual dari telur ayam organik. Kemudian dengan metode yang cepat dan akurat serta tidak memerlukan biaya dan persiapan sampel yang berbahaya.



% Baris ini digunakan untuk membantu dalam melakukan sitasi
% Karena diapit dengan comment, maka baris ini akan diabaikan
% oleh compiler LaTeX.
\begin{comment}
\bibliography{daftar-pustaka}
\end{comment}
