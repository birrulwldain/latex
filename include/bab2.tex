%-------------------------------------------------------------------------------
%                            BAB II
%               TINJAUAN PUSTAKA DAN DASAR TEORI
%-------------------------------------------------------------------------------
\chapter{TINJAUAN PUSTAKA}
\par Dalam bab ini, dibahas berbagai teori yang relevan dan penelitian terdahulu yang mendukung kajian ini. Pembahasan meliputi teknik spektroskopi yang digunakan, metode analisis spektrum, serta penerapan teknik statistik dalam pemrosesan data spektroskopi terutama untuk penerapan \textit{Deep Learning} didalamnya. Penggunaan arsitektur \textit{Convolutional Neural Networks} (CNN) digunakan untuk mengolah dari data mentah menjadi bla bla bla \cite{Zhao2022}.

\section{Laser Induced Breakdown Spectroscopy (LIBS)}
\par \textit{Laser Induced Breakdown Spectroscopy} (LIBS) merupakan teknik spektroskopi yang memanfaatkan plasma untuk menganalisis komposisi elemen dari berbagai jenis material. Proses ini dimulai dengan tembakan laser berenergi tinggi ke sampel, yang menyebabkan sampel tersebut menguap dan membentuk plasma. Plasma yang terbentuk kemudian memancarkan cahaya yang dapat dianalisis untuk menentukan elemen yang terkandung dalam sampel \citep{morris2020}.

\par Salah satu keunggulan utama LIBS adalah kemampuannya untuk melakukan analisis tanpa persiapan sampel yang rumit. Dengan teknik ini, analisis dapat dilakukan secara langsung pada sampel dalam bentuk asli tanpa perlu proses pengolahan yang ekstensif. Selain itu, LIBS dapat diterapkan pada berbagai jenis material, mulai dari logam dan mineral hingga bahan biologis \citep{harrison2021}.

\par Namun, meskipun LIBS memiliki banyak kelebihan, teknik ini juga memiliki keterbatasan. Resolusi spektrum yang dihasilkan bisa terbatas, dan hasil analisis bisa dipengaruhi oleh kondisi lingkungan seperti suhu dan tekanan. Untuk mengatasi masalah ini, metode kalibrasi yang akurat dan teknik pemrosesan data yang cermat diperlukan. Kalibrasi yang tepat membantu memastikan bahwa hasil analisis LIBS akurat, sementara pemrosesan data yang baik memungkinkan ekstraksi informasi yang lebih relevan dari spektrum yang diukur \citep{corsi2021}.

\section{Profil Voigt dalam Analisis Spektrum}
\par Dalam analisis spektrum, terutama dalam konteks LIBS, \textit{profil Voigt} merupakan alat yang penting untuk memodelkan bentuk garis spektrum. Profil Voigt menggabungkan efek dari lebar garis Lorentzian yang disebabkan oleh efek tekanan dan lebar Gaussian yang disebabkan oleh efek Doppler. Kombinasi ini memberikan gambaran yang lebih realistis tentang bentuk garis spektrum yang diukur \citep{voigt1929}.

\par Fungsi Voigt merupakan hasil konvolusi dari fungsi Lorentzian dan Gaussian. Konvolusi ini mencerminkan pengaruh berbagai faktor yang mempengaruhi lebar garis spektrum. Penggunaan profil Voigt dalam analisis spektrum sangat berguna untuk mendapatkan informasi yang lebih akurat mengenai karakteristik spektral dari elemen yang dianalisis. Misalnya, dalam LIBS, lebar garis spektrum dapat memberikan informasi penting mengenai komposisi dan konsentrasi elemen dalam sampel \citep{jeong2020}.

\par Implementasi profil Voigt dalam analisis spektrum LIBS biasanya melibatkan perangkat lunak pemodelan spektrum. Perangkat lunak ini membantu dalam fitting data spektrum dengan model Voigt, yang memungkinkan peneliti untuk menentukan parameter spektral seperti lebar garis dan intensitas puncak dengan akurasi yang lebih tinggi. Teknik fitting ini juga membantu dalam mengurangi ketidakpastian yang mungkin timbul dari pengukuran spektrum, meningkatkan keandalan hasil analisis \citep{smith2020}.

\section{Metode Principal Component Analysis (PCA)}
\par \textit{Principal Component Analysis} (PCA) adalah metode statistik yang digunakan untuk mereduksi dimensi data dengan mengubah variabel yang berkorelasi menjadi variabel yang tidak berkorelasi, yang disebut komponen utama. Teknik ini berguna untuk mengidentifikasi pola dalam data dan mengurangi jumlah dimensi yang diperlukan untuk analisis, sambil tetap mempertahankan informasi penting dalam data \citep{jolliffe2002}.

\par Dalam analisis LIBS, PCA digunakan untuk mengelompokkan data spektrum berdasarkan kesamaan karakteristik spektral. Dengan menggunakan PCA, peneliti dapat mengidentifikasi fitur utama dalam spektrum yang mungkin tidak langsung terlihat dari data mentah. PCA juga membantu dalam visualisasi data, memungkinkan peneliti untuk melihat pola dan hubungan antara berbagai komponen spektral \citep{martin2020}.

\par Selain itu, PCA berguna untuk mengurangi noise dan meningkatkan kualitas data yang dianalisis. Dengan mengekstraksi komponen utama yang paling signifikan, PCA dapat mengeliminasi variabel yang kurang penting, meningkatkan akurasi dan efisiensi proses analisis. Penerapan PCA dalam LIBS memungkinkan peneliti untuk memperoleh wawasan yang lebih mendalam mengenai karakteristik spektral dan hubungan antara berbagai elemen dalam sampel \citep{lee2021}.

\section{Profil Voigt dalam Analisis Spektrum}
\par Profil Voigt adalah gabungan dari dua fungsi profil spektral: profil Lorentzian dan Gaussian. Fungsi Voigt sering digunakan untuk mendeskripsikan bentuk garis spektrum yang diukur dalam berbagai teknik spektroskopi, termasuk Laser Induced Breakdown Spectroscopy (LIBS). Fungsi Voigt dapat didefinisikan sebagai konvolusi dari fungsi Lorentzian dan Gaussian.

\subsection{Profil Lorentzian}
\par Fungsi Lorentzian, \( L(\lambda) \), menggambarkan lebar garis spektrum yang disebabkan oleh efek tekanan atau damping. Dalam konteks panjang gelombang \( \lambda \), fungsi Lorentzian dapat dinyatakan sebagai:
\begin{equation}
L(\lambda) = \frac{\Gamma / 2\pi}{(\lambda - \lambda_0)^2 + (\Gamma / 2)^2}
\end{equation}
di mana \( \lambda \) adalah panjang gelombang yang diukur, \( \lambda_0 \) adalah panjang gelombang pusat dari garis spektrum, dan \( \Gamma \) adalah lebar garis Lorentzian yang berhubungan dengan lebar dari garis spektrum akibat efek tekanan. Dalam hal ini, \( \Gamma \) adalah Full Width at Half Maximum (FWHM) dari profil Lorentzian.

\subsection{Profil Gaussian}
\par Fungsi Gaussian, \( G(\lambda) \), menggambarkan lebar garis spektrum yang disebabkan oleh efek Doppler, yang terkait dengan pergerakan relatif antara sumber spektrum dan detektor. Fungsi Gaussian dalam konteks panjang gelombang \( \lambda \) dinyatakan sebagai:
\begin{equation}
G(\lambda) = \frac{1}{\sqrt{2\pi \sigma^2}} \exp\left(-\frac{(\lambda - \lambda_0)^2}{2\sigma^2}\right)
\end{equation}
di mana \( \sigma \) adalah deviasi standar dari distribusi Gaussian, yang berkaitan dengan lebar dari garis spektrum dalam konteks efek Doppler. Deviasi standar \( \sigma \) berhubungan dengan Half Width at Half Maximum (HWHM) Gaussian, yang dinyatakan sebagai \( b = \sigma \sqrt{2 \ln 2} \).

\subsection{Konvolusi Gaussian dan Lorentzian}
\par Fungsi Voigt, \( V(\lambda; \Gamma, \sigma) \), adalah hasil konvolusi dari fungsi Lorentzian dan Gaussian. Konvolusi ini menghasilkan profil spektrum yang menggabungkan kedua efek. Fungsi Voigt didefinisikan sebagai:
\begin{equation}
V(\lambda; \Gamma, \sigma) = \int_{-\infty}^{\infty} G(\lambda - \lambda') L(\lambda') \, d\lambda'
\end{equation}
di mana \( \Gamma \) adalah lebar garis Lorentzian (FWHM), dan \( \sigma \) adalah deviasi standar Gaussian (HWHM).

\par Fungsi Voigt dapat disederhanakan dengan menggunakan fungsi \textit{W} (fungsi Voigt) yang merupakan integral konvolusi dari fungsi Lorentzian dan Gaussian. Fungsi Voigt dapat dinyatakan sebagai: \cite{Godio2016}
\begin{equation}
V(\lambda; a, b) = \text{Re} \left[ W\left(\frac{\lambda - \lambda_0 + i a}{b}\right) \right]
\end{equation}
di mana \( a \) adalah lebar Lorentzian (FWHM) dan \( b \) adalah deviasi standar Gaussian (HWHM). Fungsi Voigt kompleks, \( W(z) \), didefinisikan sebagai:
\begin{equation}
W(z) = e^{-z^2} \left( 1 + \text{erfi}(z) \right)
\end{equation}
dengan \text{erfi}(z) sebagai fungsi kesalahan kompleks.